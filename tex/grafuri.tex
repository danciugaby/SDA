\chapter{Grafuri}

Acest capitol introduce conceptul de graf, care este structura de date fundamentală pentru majoritatea algoritmilor studiați în continuare.

\section{Noțiuni ajutătoare}

\subsection{Definiție}

Un graf este o pereche notată $G=<V,E>$ unde $V$ este mulțimea de vârfuri sau noduri, iar $E\subseteq V \times V$ este mulțimea de muchii.

O muchie de la vârful $a$ la vârful $b$ este notată cu perechea ordonată $(a,b)$. În unele grafuri denumite \textit{grafuri ponderate}, acestei perechi ii se poate asocia o valoare ce poartă denumirea de costul muchiei.

\begin{figure}[H] 
	\centering	
	{
		\Image[width=0.5\textwidth]{grafuri/simplegraph}
	}
	\caption{Exemplu de graf.} 
	\label{fig:simplegraph}
\end{figure}

În graful din figura \ref{fig:simplegraph} $G=(V,E)$ cu $V={1,2,3,4,5,6}$ și $E=(1,2),(2,1),(1,4),(4,1),(2,5),(5,2),(4,5),(5,4),(2,6),(6,2),(3,6),(6,3)$.

În cele ce urmează vom defini câteva noțiuni de bază ce vor ajuta în expunerile ulterioare ale algoritmilor ce se bazează pe grafuri.

Un graf orientat sau un digraf, este o pereche notată tot $G=<V,E>$  în care fiecare muchie este o perechea notată ${u,v}$ unde muchia formată între nodurile $u,v$ pleacă din $u$ şi ajunge în $v$. În acest tip de graf, este permisă muchia ce are acelaşi nod ca destinaţie şi sursă. În figura \ref{fig:ograf} este prezentat un graf orientat.

\begin{figure}[H] 
	 \begin{subfigure}{0.5\textwidth}
		\Image[width=\textwidth]{/grafuri/orientedgraph}
		\caption{Graf orientat}
		\label{fig:ograf}
	\end{subfigure}
	\begin{subfigure}{0.5\textwidth}
		\Image[width=0.8\textwidth]{/grafuri/subgraph}
		\caption{Subgraf obținut din graful orientat}
		\label{fig:subgraf}
	\end{subfigure}
	\caption{Exemplu de graf orientat.} 
	\label{fig:orientedgraph}
\end{figure}

Un \textit{subgraf} este un graf $G_{s}=<V^{'},E^{'}>$, unde $V^{'}\subseteq V$ iar $E^{'}\subseteq E$, adică se mențin doar acele muchii ce unesc noduri din $V^{'}$. Un exemplu de subgraf se poate observa în figura \ref{fig:subgraf} unde s-au păstrat nodurile 1 și 2 cu muchiile atașate acestor noduri.

Un \textit{graf parțial} este un graf $G_{p}=<V,E^{''}>$ unde  $E^{''}\subseteq E$. Astfel se păstrează numărul de noduri din graful inițial, dar se elimină una sau mai multe muchii. Un exemplu de astfel de graf se găsește în figura \ref{fig:partialgraph}. Acest graf a fost obținut plecând de la graful din figura \ref{fig:ograf}.

\begin{figure}[H] 
	\centering	
	{
		\Image[width=0.4\textwidth]{grafuri/partialgraph}
	}
	\caption{Exemplu de graf parțial.} 
	\label{fig:partialgraph}
\end{figure}

Un graf \textit{neorientat} este un graf $G=<V,E>$ în care ordinea dispunerilor nodurilor în pereche nu contează. O muchie va fi formată din mulțimea ${u,v}$ unde $u,v \in V$ și $u \neq v$. Prin convenție vom folosin notația $(u,v)$ pentru a delimita o muchie dintr-un graf neorientat. Un astfel de graf este reprezentat în figura \ref{fig:simplegraph}.

Două noduri $a$ și $b$ sunt \textit{adiacente} dacă sunt unite print-o muchie. 

Un $drum$ este o succesiune de muchii de forma:

$(a_1,a_2), (a_2,a_3), ..., (a_{n-1},a_n)$

Un exemplu de astfel de drum se găsește în figura \ref{fig:drum} trasat cu roșu. Drumul poate fi exprimat astfel: (1,2), (2,5), (5,4).

\begin{figure}[H] 
	\centering	
	{
		\Image[width=0.3\textwidth]{grafuri/drum}
	}
	\caption{Exemplu de graf parțial.} 
	\label{fig:drum}
\end{figure}

\textit{Lungimea drumului} este egală cu numărul muchiilor din care este alcătuit.

Un \textit{drum simplu} este acel drum în care niciun vârf nu se repetă.

Un \textit{ciclu} este un drum simplu, cu excepția primului și ultimului vârf care sunt unul și același.

Un \textit{graf aciclic} este un graf ce nu conșine niciun ciclu.

Un graf neorientat este $conex$ dacă între oricare două vârfuri există un drum. Pentru grafuri orientate, această opțiune este întărită: un graf este $tare$ conex dacă între oricare două noduri $a$ și $b$ există cel puțin un drum de la $a$ la $b$ dar și de la $b$ la $a$.

În figura \ref{fig:conex} avem un exemplu de graf conex.

\begin{figure}[H] 
	\centering	
	{
		\Image[width=0.3\textwidth]{grafuri/conex}
	}
	\caption{Exemplu de graf neorientat conex.} 
	\label{fig:conex}
\end{figure}

O \textit{componentă conexă} este un subgraf conex minimal, adică un subgraf conex în care niciun vârf din subgraf nu este adiacent cu niciun alt vârf din exteriorul subgrafului.
Împărțirea unui graf $G=<V,E>$ în componentele sașe conexe determină o partiție a lui $V$ și una a lui $E$. În figura \ref{fig:conexcomp} avem o reprezentare a unor componente conexe.

\begin{figure}[H] 
	\centering	
	{
		\Image[width=0.5\textwidth]{grafuri/conexcomp}
	}
	\caption{Exemplu de graf cu 2 componente conexe: {1,5,6} și {2,3,4}.} 
	\label{fig:conexcomp}
\end{figure}

Vârfurilor unui graf neorientat li se pot atașa informații numite uneori \textit{valori}, iar muchiilor li se pot atașa informații numite \textit{lungimi} sau \textit{costuri}.

\section{Reprezentarea grafurilor}

Există mai multe moduri de a reprezenta un graf, fiecare având avantaje și dezavantaje. Există mai multe criterii conform cărora se alege tipul de reprezentare:

\begin{itemize}
\item {memoria necesară alocării grafului}
\item {timpul necesar accesării unei muchii din graf}
\item {timpul necesar accesării vecinilor unui nod}
\end{itemize}



\begin{figure}[H] 
	\centering	
	{
		\Image[width=0.6\textwidth]{grafuri/reprezentare}
	}
	\caption{Graf neorientat.} 
	\label{fig:reprezentare}
\end{figure}
Având ca exemplu graful din figura \ref{fig:reprezentare} vom analiza fiecare tip de reprezentare a grafurilor.

\subsection{Matricea de adiacență}

Se numește matrice de adiacență $A$, o matrice de $\left |  V \right |\times \left |  V \right |$ în care $A[i,j]=true$ dacă nodurile $i$ și $j$ sunt adiacente și $A[i,j]=false$ în caz contrar. O variantă alternativă ar fi să atribuim lui $A[i,j]$ costul muchiei dintre $i$ și $j$ considerând $A[i,j]=+\infty$ atunci când cele două vârfuri nu sunt adiacente. Pentru graful din figura \ref{fig:reprezentare} matricea de adiacență este cea din tabelul \ref{table:adiacenta}.


\begin{table}[h]
	\centering
	\begin{tabular}{c c c c c c}
		0 & 1 & 0 & 0 & 0 & 1\\ 
		1 & 0 & 1 & 0 & 1 & 1\\ 
		0 & 1 & 1 & 1 & 0 & 0\\ 
		0 & 0 & 1 & 0 & 1 & 0\\ 
		0 & 1 & 0 & 1 & 0 & 0\\ 
		1 & 1 & 0 & 0 & 0 & 0\\ 
		
	\end{tabular}
	\caption{Reprezentarea unei matrice de adiacență}
	\label{table:adiacenta}
\end{table}

Cu ajutorul matricilor de adiacență putem afla în timp constant dacă o muchie există sau nu în graf. Este suficient să interogăm $A[i,j]$ și vom afla dacă muchia între $i$ și $j$ există. 

Există totuși două dezavantaje ale acestei reprezentări. În primul rând spațiul alocat pentru a reprezenta muchiile grafului este $\Theta(V^2)$ chiar dacă graful conține mai puțin de $\left |  V \right | $ muchii.
În al doilea rând, pentru a interoga toți vecinii nodului $i$ va trebui parcursă întreaga linie $i$, pentru a interoga toate cele $\left |  V \right | $ noduri, chiar dacă $i$ are un singur vecin.

\subsection{Liste de muchii}

O listă de muchii este un șir de perechi ordonate, fiecare pereche sau tuplu reprezentând o muchie. Dacă unei muchii ii se asociază un cost, atunci perechea devine un triplu format din nodul $i$, nodul $j$ si costul $c$.

Pentru graful din figura \ref{fig:reprezentare}, lista de muchii este următoarea:

{ (1,2,$e_1$),(2,3,$e_2$),(3,3,$e_8$),(3,4,$e_3$),(4,5,$e_4$),(5,2,$e_5$),(6,2,$e_6$),(1,6,$e_7$)}

unde $e_{1..8}$ reprezintă costul unei muchii adică un număr întreg.

Aceste liste sunt simplu de alcătuit și spațiul necesar alocării lor este liniar: $\Theta(E)$, dar căutarea unei anumite muchii ce leagă nodurile $i$ și $j$ necesită o parcurgere liniară a $\left |  E \right | $ muchii.

\subsection{Liste de adiacență}

Într-o listă de adiacență fiecare nod $i$ conține un șir de noduri adiacent lui. Astfel avem o listă sau un șir de $\left |  V \right | $ liste de adiacență precum cea din figura \ref{fig:adiacenta}.


\begin{figure}[H] 
	\centering	
	{
		\Image[width=0.6\textwidth]{grafuri/adiacenta}
	}
	\caption{Liste de adiacență} 
	\label{fig:adiacenta}
\end{figure}

Putem accesa lista oricărui nod într-un timp constant. Pentru a afla dacă muchia $(i,j)$ se află sau nu în graf, interogăm lista de adiacență a lui $i$ și căutăm în această listă nodul $j$. Timpul necesar acestei căutări este $\Theta(d)$ unde $d$ este \textit{gradul} lui $i$, adică numărul total de vecini al lui $i$.

Ca și memorie o listă va ocupa cel puțin  $\left |  V \right | $ poziții în cazul în care toate nodurile sunt \textit{izolate} (adică nu au niciun vecin). Dacă fiecare nod are un număr maxim de vecini ($\left |  V \right | -1 $) atunci memoria alocată va fi în $\Theta( V^2 )$.


\section{Aplicații ale grafurilor}

Teoria grafurilor are aplicații în diferite domenii dintre care iată câteva:


\begin{itemize}
	\item {modelarea frecvețelor de tact în circuite digitale}
	\item {în construcții pentru estimarea cablajului necesar}
	\item {la rutarea semnalelor în rețele predefinite}
	\item {la estimarea costurilor drumurilor în hărți}
	\item {la evaluarea și predicția evenimentelor economice}
	\item {la efectuarea diferitelor simulări de la dinamica consumului până la simularea traficului urban}	
\end{itemize}

Acestea sunt doar câteva aplicații concrete ale teoriei grafurilor. 
Evident lista este mult mai lungă, în continuarea acestui curs urmând să descoperim și alte potențiale utilizări ale algoritmilor ce utilizează grafuri.
