\chapter{Grafuri}

Acest capitol introduce conceptul de graf, care este structura de date fundamentală pentru majoritatea algoritmilor studiați în continuare.

\section{Graful și noțiuni ajutătoare}

\subsection{Definiție}

Un graf este o pereche notată $G=<V,E>$ unde $V$ este mulțimea de vârfuri sau noduri, iar $E\subseteq V \times V$ este mulțimea de muchii.

O muchie de la vârful $a$ la vârful $b$ este notată cu perechea ordonată $(a,b)$. În unele grafuri acestei perechi ii se poate asocia o valoare ce poartă denumirea de costul muchiei.

\begin{figure}[H] 
	\centering	
	{
		\Image[width=0.8\textwidth]{grafuri/simplegraph}
	}
	\caption{Exemplu de graf.} 
	\label{fig:simplegraph}
\end{figure}

În graful din figura \ref{fig:simplegraph} $G=(V,E)$ cu $V={1,2,3,4,5,6}$ și $E=(1,2),(2,1),(1,4),(4,1),(2,5),(5,2),(4,5),(5,4),(2,6),(6,2),(3,6),(6,3)$.

